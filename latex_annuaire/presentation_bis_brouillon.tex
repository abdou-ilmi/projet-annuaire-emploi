\documentclass{beamer}
\usepackage[utf8]{inputenc}
\usetheme{Madrid}
\usepackage{graphicx}
\title[Annuaire Statistique de l'Emploi]{Processus de Production de l’Annuaire Statistique de l’Emploi 2025 – Atelier de Formation}
\author{ONEQ -- Projet SKILLS}
\date{Juin 2025}

\begin{document}

\begin{frame}
  \titlepage
\end{frame}

% Slide : Introduction
\begin{frame}{Objectifs de l'atelier}
\begin{itemize}
  \item Expliquer le processus de production de l’annuaire statistique de l’emploi
  \item Renforcer la compréhension du rôle des différentes institutions productrices de données
  \item Initier les participants à la conception d’une fiche de collecte et à l’élaboration de tableaux statistiques
\end{itemize}
\end{frame}

% Slide : Pourquoi un Annuaire Statistique de l’Emploi ?
\begin{frame}{Pourquoi un Annuaire Statistique de l’Emploi ?}
\begin{itemize}
  \item Un outil de référence pour la prise de décision en matière d’emploi, formation et développement économique
  \item Permet de suivre les tendances du marché du travail : emploi, chômage, qualité de l’emploi, insertion, vulnérabilités
  \item Fournit une base factuelle pour les politiques publiques et les investissements
  \item Améliore la transparence et la coordination entre institutions productrices de données
\end{itemize}
\end{frame}

% Slide : Étapes principales du processus
\begin{frame}{Étapes principales de production}
\begin{enumerate}
  \item Cadrage et état des lieux
  \item Conception des outils de collecte
  \item Collecte des données auprès des institutions
  \item Traitement et structuration dans une base unique
  \item Analyse et production des tableaux/statistiques
  \item Mise en forme graphique et publication
\end{enumerate}
\end{frame}

% Slide : Les institutions impliquées
\begin{frame}{Les institutions impliquées}
\begin{itemize}
  \item \textbf{INSTAD} : Données des enquêtes emploi, recensement, EDAM
  \item \textbf{ANEFIP} : Données sur les chercheurs d’emploi, offres, placements
  \item \textbf{CNSS} : Données sur les salariés, affiliations, salaires, accidents
  \item \textbf{ADDS, ANPH, ONARS, Ministère Femme/Famille, MENFOP, ANPI} : Données complémentaires selon leur domaine (vulnérabilité, formation, investissement…)
\end{itemize}
\end{frame}

% Slide : Atelier 1
\begin{frame}{Atelier 1 -- Conception d’une fiche de collecte}
\begin{itemize}
  \item Présentation d’un organisme fictif ou réel
  \item Objectif : réfléchir aux données utiles pour l’annuaire
  \item Élaboration collective d’une fiche de collecte : quelles données, quelles périodes, quelles unités ?
\end{itemize}
\end{frame}

% Slide : Atelier 2
\begin{frame}{Atelier 2 -- À partir d’une fiche remplie}
\begin{itemize}
  \item Objectif : traduire une fiche en tableaux statistiques exploitables
  \item Répartir les données en tableaux par sexe, tranche d’âge, secteur, etc.
  \item Réflexion sur les indicateurs clés à en tirer
\end{itemize}
\end{frame}

% Slide : Clôture
\begin{frame}{Conclusion}
\begin{itemize}
  \item L’annuaire statistique est un outil stratégique de pilotage du marché du travail
  \item Le rôle de chacun est essentiel pour garantir la qualité des données produites
  \item Cette formation vise à renforcer la capacité de l’équipe ONEQ à produire un document robuste, utile et durable
\end{itemize}
\end{frame}

\end{document}
