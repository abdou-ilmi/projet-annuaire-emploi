% Version avec notes pour l’animateur
\documentclass{beamer}
\usetheme{Madrid}
\usepackage[utf8]{inputenc}
\usepackage[french]{babel}
\title{Atelier de conception de fiches de collecte}
\author{ONEQ - Projet SKILLS}
\date{Djibouti, 2025}

\begin{document}

\frame{\titlepage}

\begin{frame}{Question 1 : Nom de l’institution ciblée}
\pause
\textbf{Question à discuter :}
\begin{itemize}
  \item Quel est le nom de l’institution ciblée ?
\end{itemize}
\note{
Cette question sert à identifier clairement l’organisme concerné. Exemples : ANEFIP, CNSS, ANPH, MENFOP, etc.
}
\end{frame}

\begin{frame}{Question 2 : Mission principale}
\pause
\textbf{Question à discuter :}
\begin{itemize}
  \item Quelle est sa mission principale en lien avec l’emploi ?
\end{itemize}
\note{
Cela permet de comprendre en quoi les activités de l’institution produisent ou influencent des données liées à l’emploi (ex : placement, formation, inclusion…)
}
\end{frame}

\begin{frame}{Question 3 : Données produites régulièrement}
\pause
\textbf{Question à discuter :}
\begin{itemize}
  \item Quelles sont les données qu’elle produit régulièrement ?
\end{itemize}
\note{
Exemples : nombre de bénéficiaires formés, placements, inscriptions, salariés déclarés… Il faut cerner les séries régulières.
}
\end{frame}

\begin{frame}{Question 4 : Données utiles pour l’annuaire}
\pause
\textbf{Question à discuter :}
\begin{itemize}
  \item Quelles données pourraient être utiles pour l’annuaire ?
\end{itemize}
\note{
Parmi toutes les données produites, lesquelles sont stratégiques pour renseigner les tableaux du marché du travail ?
}
\end{frame}

\begin{frame}{Question 5 : Format des données}
\pause
\textbf{Question à discuter :}
\begin{itemize}
  \item Sous quelle forme les données sont-elles disponibles (Excel, base de données, rapports, etc.) ?
\end{itemize}
\note{
Cela permet d’évaluer la facilité d’intégration dans la base de données de l’annuaire (saisie manuelle vs import automatisé).
}
\end{frame}

\begin{frame}{Question 6 : Désagrégation des données}
\pause
\textbf{Question à discuter :}
\begin{itemize}
  \item Ces données sont-elles désagrégées par sexe, âge, région, type d’emploi ?
\end{itemize}
\note{
Les désagrégations sont essentielles pour produire des tableaux croisés et des indicateurs ciblés.
}
\end{frame}

\begin{frame}{Question 7 : Fréquence de production}
\pause
\textbf{Question à discuter :}
\begin{itemize}
  \item À quelle fréquence sont-elles produites (mensuelle, annuelle…) ?
\end{itemize}
\note{
Cela permet de déterminer si les données seront disponibles à temps pour l’annuaire annuel.
}
\end{frame}

\begin{frame}{Question 8 : Organisation de la base de données}
\pause
\textbf{Question à discuter :}
\begin{itemize}
  \item Existe-t-il une base centrale ou des fichiers multiples ?
\end{itemize}
\note{
Cela indique si la donnée est centralisée, ce qui facilite la récupération et le traitement, ou dispersée, ce qui complexifie la collecte.
}
\end{frame}

\end{document}
