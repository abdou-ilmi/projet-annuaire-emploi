\documentclass{beamer}
\usetheme{Madrid}
\usepackage[utf8]{inputenc}
\usepackage[french]{babel}
\title{Processus de production de l'Annuaire Statistique de l'Emploi 2025}
\author{ONEQ - Projet SKILLS}
\date{Atelier de renforcement de capacités\\Djibouti, 2025}

\begin{document}

\frame{\titlepage}

\begin{frame}{Objectifs de l'atelier}
\begin{itemize}
    \item Comprendre les étapes de production de l'annuaire statistique
    \item Identifier les sources et fiches de collecte utilisées
    \item S'exercer à concevoir une fiche de collecte pertinente
    \item Déduire les tableaux statistiques à partir d'une fiche donnée
\end{itemize}
\end{frame}

\begin{frame}{Étapes de production de l'annuaire (selon le cadrage)}
\begin{enumerate}
    \item État des lieux des producteurs de données
    \item Conception de la base de données et des fiches
    \item Collecte des données auprès des institutions
    \item Traitement, nettoyage, structuration
    \item Production des tableaux et graphiques
    \item Rédaction et mise en page
    \item Validation et diffusion
\end{enumerate}
\end{frame}

\begin{frame}{Phase 1 : État des lieux}
\begin{itemize}
    \item Identification des acteurs (INSTAD, ANEFIP, CNSS...)
    \item Analyse des flux de données existants
    \item Élaboration de fiches de collecte spécifiques
    \item Synthèse dans un rapport d'état des lieux
\end{itemize}
\end{frame}

\begin{frame}{Phase 2 : Conception de la base de données}
\begin{itemize}
    \item Normalisation des indicateurs
    \item Définition des tables, champs, formats
    \item Structuration pour exploitation semi-automatique
\end{itemize}
\end{frame}

\begin{frame}{Phase 3 : Collecte des données}
\begin{itemize}
    \item Envoi des fiches de collecte aux institutions
    \item Appui si besoin (explication, remplissage)
    \item Suivi et relances
\end{itemize}
\end{frame}

\begin{frame}{Phase 4 : Traitement et production}
\begin{itemize}
    \item Vérification de cohérence et de complétude
    \item Agrégation, calculs, désagrégations
    \item Conception des tableaux et graphiques
\end{itemize}
\end{frame}

\begin{frame}{Phase 5 : Finalisation et diffusion}
\begin{itemize}
    \item Mise en forme en LaTeX ou Word
    \item Relecture et validation
    \item Impression ou diffusion numérique
\end{itemize}
\end{frame}

\begin{frame}{Atelier pratique : conception de fiche}
\begin{enumerate}
    \item En groupe : choisissez un organisme (ANPH, CNSS...)
    \item Listez les données utiles à collecter
    \item Remplissez la fiche de collecte (modèle fourni)
\end{enumerate}
\end{frame}

\begin{frame}{Atelier pratique : déduction des tableaux}
\begin{enumerate}
    \item À partir d'une fiche remplie (ex. ANPH), proposez :
    \begin{itemize}
        \item Les tableaux statistiques qu'on peut produire
        \item Leurs colonnes, lignes, désagrégations (âge, sexe...)
        \item Types de graphiques associés
    \end{itemize}
\end{enumerate}
\end{frame}

\begin{frame}{Merci !}
\centering
Questions ?\\
\vspace{0.5cm}
\includegraphics[width=2cm]{example-image-a}
\end{frame}

\end{document}
