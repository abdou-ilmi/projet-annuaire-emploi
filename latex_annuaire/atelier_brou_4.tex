\documentclass{beamer}
\usetheme{Madrid}
\usepackage[utf8]{inputenc}
\usepackage[french]{babel}

\title{Atelier 2 -- Lecture d'une fiche remplie}
\author{ONEQ -- Projet SKILLS}
\date{Juin 2025}

\begin{document}

\frame{\titlepage}

% Slide Questions
\begin{frame}{Atelier 2 -- Questions de lecture d'une fiche remplie}
\begin{itemize}
  \item Quelles sont les variables disponibles dans la fiche ?
  \item Peut-on croiser ces variables entre elles ? Si oui, lesquelles ?
  \item Peut-on d\'esagr\'eger par sexe, \^age, ou autre caract\'eristique ?
  \item Quels types de tableaux peut-on construire ? (R\'epartition, \'evolution, parts, etc.)
  \item Peut-on calculer des taux, des moyennes, des indicateurs synth\'etiques ?
  \item Quelle visualisation serait adapt\'ee (barres, histogrammes, courbes, camemberts...) ?
  \item Quelles limites ou biais \'eventuels \`a mentionner avec ces donn\'ees ?
\end{itemize}
\end{frame}

% Notes pour l'animateur
\note{
\textbf{D\'etails explicatifs pour l'animateur :}

\textbf{1. Variables disponibles :} Lister les champs dans la fiche : sexe, \^age, type de programme, situation avant/apr\`es, etc.

\textbf{2. Croisements possibles :} Ex : sexe \texttimes{} \^age, formation \texttimes{} insertion, r\'egion \texttimes{} genre...

\textbf{3. D\'esagr\'egation :} Est-ce que les donn\'ees permettent des regroupements plus fins ? (Oui/Non et lesquels ?)

\textbf{4. Types de tableaux :} Effectifs, parts, comparaisons, r\'epartitions, \'evolutions temporelles...

\textbf{5. Indicateurs calculables :} Taux d'insertion = ins\'er\'es / form\'es ; moyennes d'\^age ; proportions de femmes...

\textbf{6. Visualisations :} Barres, courbes, camemberts, selon nature des variables.

\textbf{7. Limites/biais :} Agr\'egations trop fortes, absence de certaines variables, couverture partielle, etc.
}

\end{document}
