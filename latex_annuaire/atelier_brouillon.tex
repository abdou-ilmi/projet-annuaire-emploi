\documentclass{beamer}
\usepackage[utf8]{inputenc}
\usepackage[french]{babel}
\usepackage{graphicx}
\usetheme{Madrid}

\title[Annuaire Statistique de l'Emploi]{Processus de production de l'Annuaire Statistique de l'Emploi 2025 -- Atelier de renforcement de capacit\'es}
\author{ONEQ -- Projet SKILLS}
\date{Djibouti, Juin 2025}

\begin{document}

\frame{\titlepage}

% Objectifs
\begin{frame}{Objectifs de l'atelier}
\begin{itemize}
  \item Comprendre les \'etapes de production de l'annuaire statistique
  \item Identifier les institutions impliqu\'ees et les fiches de collecte
  \item Renforcer la comp\'etence \`a concevoir une fiche de collecte pertinente
  \item D\'eduire les tableaux statistiques \`a partir d'une fiche donn\'ee
\end{itemize}
\end{frame}

% Pourquoi un Annuaire ?
\begin{frame}{Pourquoi un Annuaire Statistique de l’Emploi ?}
\begin{itemize}
  \item Outil de r\'ef\'erence pour les politiques publiques et l’aide \`a la d\'ecision
  \item Suivi des tendances de l’emploi, du ch\^omage et des dynamiques du march\'e du travail
  \item Am\'elioration de la coordination interinstitutionnelle et de la transparence
  \item Renforcement de la disponibilit\'e de donn\'ees fiables sur le march\'e du travail
\end{itemize}
\end{frame}

% Etapes générales
\begin{frame}{\'Etapes principales de production}
\begin{enumerate}
  \item \textbf{Cadrage et \'etat des lieux} des producteurs de donn\'ees
  \item Conception des \textbf{fiches de collecte} et de la \textbf{base de donn\'ees}
  \item \textbf{Collecte} des donn\'ees
  \item \textbf{Traitement, structuration et analyse}
  \item \textbf{Production des tableaux et graphiques}
  \item \textbf{Mise en forme, validation et diffusion}
\end{enumerate}
\end{frame}

% Institutions impliquées
\begin{frame}{Institutions partenaires principales}
\begin{itemize}
  \item \textbf{INSTAD} : enqu\^etes emploi, EDAM, recensement (chapitre 1)
  \item \textbf{ANEFIP} : chercheurs d’emploi, offres, formations (chapitres 2 et 3)
  \item \textbf{CNSS} : salari\'es, affiliations, salaires, accidents (chapitre 5)
  \item \textbf{ADDS, ANPH, ONARS, Minist\`ere Femme/Famille, MENFOP, ANPI} : donn\'ees sectorielles (chapitres 3 et 6)
\end{itemize}
\end{frame}

% Clôture
\begin{frame}{Conclusion}
\begin{itemize}
  \item L’annuaire permet de structurer les connaissances sur le march\'e du travail
  \item Le processus repose sur une bonne coordination institutionnelle
  \item La ma\^itrise des outils de collecte et de restitution est essentielle
\end{itemize}
\centering
Merci pour votre participation !
\end{frame}

\end{document}
