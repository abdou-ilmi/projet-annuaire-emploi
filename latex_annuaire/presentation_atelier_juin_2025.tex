\documentclass{beamer}
\usepackage[utf8]{inputenc}
\usepackage[french]{babel}
\usepackage{graphicx}
\usetheme{Madrid}

\title[Annuaire Statistique de l'Emploi]{Processus de production de l'Annuaire Statistique de l'Emploi 2025 -- Atelier de renforcement de capacit\'es}
\author{ONEQ -- Projet SKILLS}
\date{Djibouti, Juin 2025}

\begin{document}

\frame{\titlepage}

% Objectifs
\begin{frame}{Objectifs de l'atelier}
\begin{itemize}
  \item Comprendre les \'etapes de production de l'annuaire statistique
  \item Identifier les institutions impliqu\'ees et les fiches de collecte
  \item S'exercer à concevoir une fiche de collecte pertinente
  \item D\'eduire les tableaux statistiques \`a partir d'une fiche donn\'ee
\end{itemize}
\end{frame}
%%%%%%%%%%%%%%%%%%

%\begin{itemize}
%    \item Comprendre les étapes de production de l'annuaire statistique
%    \item Identifier les sources et fiches de collecte utilisées
%    \item S'exercer à concevoir une fiche de collecte pertinente
%    \item Déduire les tableaux statistiques à partir d'une fiche donnée
%\end{itemize}
%\end{frame}

%%%%%%%%%%%%%%%%%%%%
% Pourquoi un Annuaire ?
\begin{frame}{Pourquoi un Annuaire Statistique de l’Emploi ?}
\begin{itemize}
  \item Outil de r\'ef\'erence pour les politiques publiques et l’aide \`a la d\'ecision
  \item Suivi des tendances de l’emploi, du ch\^omage et des dynamiques du march\'e du travail
  \item Am\'elioration de la coordination interinstitutionnelle et de la transparence
  \item Renforcement de la disponibilit\'e de donn\'ees fiables sur le march\'e du travail
\end{itemize}
\end{frame}

%% Etapes générales
%\begin{frame}{\'Etapes principales de production}
%\begin{enumerate}
%  \item \textbf{Cadrage et \'etat des lieux} : Cette phase consiste à identifier les institutions sources. Elle permet de structurer l’annuaire en définissant les chapitres à produire. :
%    \begin{itemize}
%      \item Chapitre 1 : Indicateurs globaux (INSTAD)
%      \item Chapitre 2 : ANEFIP (demande d’emploi)
%      \item Chapitre 3 : Programmes (ANEFIP, ADDS, ONARS, ANPH, etc.)
%      \item Chapitre 4 : Formation professionnelle (MENFOP)
%      \item Chapitre 5 : Emploi salarié formel (CNSS)
%      \item Chapitre 6 : Investissements créateurs d’emploi (ANPI)
%      \item Chapitre 7 : Emploi et migration (ONARS)
%    \end{itemize}
%  \item \textbf{Conception des fiches et de la base de donn\'ees} : Cr\'eation de fiches types pour chaque institution. Mod\'elisation de la base selon les besoins des tableaux et chapitres.
%
%  \item \textbf{Collecte des donn\'ees} : Envoi des fiches aux institutions, relances, validation progressive.
%
%  \item \textbf{Traitement, structuration et analyse} : Nettoyage, formatage, int\'egration dans la base unique. Contr\^ole de qualit\'e. Calculs automatiques pour indicateurs (taux, moyennes, parts, etc.).
%
%  \item \textbf{Production des tableaux et graphiques} : G\'en\'eration des tableaux par chapitre. Visualisation via Python/Excel/LaTeX. Adaptation aux besoins de lecture publique ou institutionnelle.
%
%  \item \textbf{Mise en forme, validation et diffusion} : Mise en page sous LaTeX. Lecture croisée. Validation technique et institutionnelle. Impression ou mise en ligne.
%\end{enumerate}
%\end{frame}

%%%%%%%%%%%%%%%%%%%%%

% Étapes principales - Partie 1
\begin{frame}{Étapes principales de production (1/2)}
\begin{enumerate}
  \item \textbf{Cadrage et état des lieux} : Cette phase consiste à identifier les institutions sources. Elle permet de structurer l’annuaire en définissant les chapitres à produire :
    \begin{itemize}
      \item Chapitre 1 : Indicateurs globaux (INSTAD)
      \item Chapitre 2 : ANEFIP (demande d’emploi)
      \item Chapitre 3 : Programmes et projets de création d' emploi (ANEFIP, ADDS, ONARS, ANPH, etc.)
      \item Chapitre 4 : Formation technique et professionnelle (MENFOP)
      \item Chapitre 5 : Travailleurs immatriculés à la CNSS
      \item Chapitre 6 : Emploi et Investissements (ANPI)
      \item Chapitre 7 : Emploi et migration (ONARS,INSTAD)
    \end{itemize}
  \item \textbf{Conception des fiches et de la base de données} : Création de fiches types pour chaque institution. Modélisation de la base selon les besoins des tableaux et chapitres.
  
\end{enumerate}
\end{frame}

% Étapes principales - Partie 2
\begin{frame}{Étapes principales de production (2/2)}
\begin{enumerate}
  \setcounter{enumi}{3}
  \item \textbf{Collecte des données} : Envoi des fiches aux institutions, relances, validation progressive.
  \item \textbf{Traitement, structuration et analyse} : Nettoyage, formatage, intégration dans la base unique. Contrôle de qualité. Calculs automatiques pour indicateurs (taux, moyennes, parts, etc.).
  \item \textbf{Production des tableaux et graphiques} : Visualisation des données sous forme de graphiques et tableaux lisibles, adaptés aux besoins du grand public et des décideurs institutionnels
  \item \textbf{Mise en forme, validation et diffusion} : Validation technique et institutionnelle. Impression ou mise en ligne.
\end{enumerate}
\end{frame}










%%%%%%%%%%%%%%%%%%%%%%%%%%%%%%%%%%


% Institutions impliquées
\begin{frame}{Institutions partenaires principales}
\begin{itemize}
  \item \textbf{INSTAD} : enqu\^etes emploi, EDAM, recensement (chapitre 1)
  \item \textbf{ANEFIP} : chercheurs d’emploi, offres, formations (chapitres 2 et 3)
  \item \textbf{CNSS} : salari\'es, affiliations, salaires, accidents (chapitre 5)
  \item \textbf{ADDS, ANPH, ONARS, Minist\`ere Femme/Famille, MENFOP, ANPI} : donn\'ees sectorielles (chapitres 3, 4, 6 et 7)
\end{itemize}
\end{frame}

% Clôture
\begin{frame}{Conclusion}
\begin{itemize}
  \item L’annuaire permet de structurer les connaissances sur le march\'e du travail
  \item Le processus repose sur une bonne coordination institutionnelle
  \item La ma\^itrise des outils de collecte et de restitution est essentielle
\end{itemize}

\end{frame}

\begin{frame}{Conclusion}
\centering
Merci pour votre participation !
\end{frame}

\end{document}
