\documentclass{beamer}
\usepackage[utf8]{inputenc}
\usetheme{Madrid}
\usepackage{lmodern}
\usepackage[french]{babel}

\title[Ateliers sur la production de l'annuaire]{Ateliers -- Production de l'annuaire statistique de l'emploi}
\author{ONEQ / Projet SKILLS}
\date{Djibouti, juin 2025}

\begin{document}

\begin{frame}
  \titlepage
\end{frame}

% Atelier 1
\begin{frame}{Atelier 1 – Élaboration d’une fiche de collecte}
\large
\textbf{Objectif :} Concevoir une fiche de collecte adaptée à un organisme producteur de données emploi.

\pause
\textbf{Questions à discuter :}
\begin{itemize}
  \item Quel est le nom de l’institution ciblée ?
  \item Quelle est sa mission principale en lien avec l’emploi ?
  \item Quelles sont les données qu’elle produit régulièrement ?
  \item Quelles données pourraient être utiles pour l’annuaire ?
  \item Sous quelle forme les données sont-elles disponibles (Excel, base de données, rapports, etc.) ?
  \item Ces données sont-elles désagrégées par sexe, âge, région, type d’emploi ?
  \item À quelle fréquence sont-elles produites (mensuelle, annuelle…) ?
  \item Existe-t-il une base centrale ou des fichiers multiples ?
\end{itemize}
\end{frame}

% Atelier 2
\begin{frame}{Atelier 2 – De la fiche de collecte aux tableaux statistiques}
\large
\textbf{Objectif :} À partir d’une fiche déjà remplie, imaginer les tableaux possibles à produire dans l’annuaire.

\pause
\textbf{Questions à discuter :}
\begin{itemize}
  \item Quelles sont les variables disponibles dans la fiche ?
  \item Peut-on croiser ces variables entre elles ? Si oui, lesquelles ?
  \item Peut-on désagréger par sexe, âge, ou autre caractéristique ?
  \item Quels types de tableaux peut-on construire ? (Répartition, évolution, parts, etc.)
  \item Peut-on calculer des taux, des moyennes, des indicateurs synthétiques ?
  \item Quelle visualisation serait adaptée (barres, histogrammes, courbes, camemberts…) ?
  \item Quelles limites ou biais éventuels à mentionner avec ces données ?
\end{itemize}
\end{frame}

\end{document}
