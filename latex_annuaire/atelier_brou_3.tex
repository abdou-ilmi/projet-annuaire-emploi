\documentclass{beamer}
\usetheme{Madrid}
\usepackage[utf8]{inputenc}
\usepackage[french]{babel}
\usepackage{tcolorbox}
\tcbuselibrary{listings, skins, breakable}

\title{Atelier : Élaboration d'une fiche de collecte \& exploitation statistique}
\author{ONEQ -- Projet SKILLS}
\date{Juin 2025}

\begin{document}

\frame{\titlepage}

\begin{frame}{Questions à discuter (Atelier 1)}
\pause
\textbf{1. Quel est le nom de l’institution ciblée ?}
\begin{tcolorbox}[colback=blue!5!white,colframe=blue!75!black,title=Exemple de réponse]
Exemple : ANPH – Agence Nationale des Personnes Handicapées
\end{tcolorbox}

\pause
\textbf{2. Quelle est sa mission principale en lien avec l’emploi ?}
\begin{tcolorbox}[colback=blue!5!white,colframe=blue!75!black,title=Exemple de réponse]
Elle met en œuvre des programmes de formation, d'insertion professionnelle et d’accompagnement pour les personnes handicapées.
\end{tcolorbox}

\pause
\textbf{3. Quelles sont les données qu’elle produit régulièrement ?}
\begin{tcolorbox}[colback=blue!5!white,colframe=blue!75!black,title=Exemple de réponse]
Nombre de personnes formées, types de formations, nombre de bénéficiaires insérés, secteurs d’activité, aides économiques versées.
\end{tcolorbox}
\end{frame}

\begin{frame}{Suite des questions}
\pause
\textbf{4. Quelles données pourraient être utiles pour l’annuaire ?}
\begin{tcolorbox}[colback=blue!5!white,colframe=blue!75!black,title=Exemple de réponse]
Taux d’insertion post-formation, répartition des bénéficiaires par sexe, tranche d’âge, région, type de handicap, etc.
\end{tcolorbox}

\pause
\textbf{5. Sous quelle forme les données sont-elles disponibles ?}
\begin{tcolorbox}[colback=blue!5!white,colframe=blue!75!black,title=Exemple de réponse]
Souvent sous Excel ou dans des rapports annuels PDF ; parfois stockées dans une base de données interne ou un système d'information.
\end{tcolorbox}
\end{frame}

\begin{frame}{Dernières questions}
\pause
\textbf{6. Les données sont-elles désagrégées ?}
\begin{tcolorbox}[colback=blue!5!white,colframe=blue!75!black,title=Exemple de réponse]
Oui : par sexe, âge, région, type de formation, statut d’insertion (inséré/non inséré), type d’aide reçue…
\end{tcolorbox}

\pause
\textbf{7. À quelle fréquence sont-elles produites ?}
\begin{tcolorbox}[colback=blue!5!white,colframe=blue!75!black,title=Exemple de réponse]
Annuellement, dans le cadre du rapport d’activité ; parfois semestriellement selon le projet ou programme financé.
\end{tcolorbox}

\pause
\textbf{8. Existe-t-il une base centrale ou des fichiers multiples ?}
\begin{tcolorbox}[colback=blue!5!white,colframe=blue!75!black,title=Exemple de réponse]
Données éparses dans plusieurs fichiers ou départements. Un effort de centralisation serait utile pour harmoniser les sources.
\end{tcolorbox}
\end{frame}

\end{document}
